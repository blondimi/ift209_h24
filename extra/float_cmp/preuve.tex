\documentclass{article}

\usepackage{fontspec}
\setmainfont{XCharter}
\setmonofont[Scale=0.875]{Fira Mono}
%\usepackage[utf8]{inputenc}
\usepackage{amsmath, amsthm, amssymb}
\usepackage[french]{cleveref}
\usepackage[french]{babel}
\usepackage{mathtools}
\usepackage{minted}
\usepackage[many]{tcolorbox}

\newenvironment{code}{
  \smallskip\begin{tcolorbox}[colback=lightgray!8, breakable, sharp corners, before upper={\parindent15pt}, parbox=false, left=0pt, right=0pt, top=-1pt, bottom=-2pt, boxrule=0.75pt, enhanced, colframe=black!25]\vspace*{-4pt}
}{\vspace*{-5pt}\end{tcolorbox}\smallskip
}

\newcommand{\N}{\mathbb{N}}
\newcommand{\R}{\mathbb{R}}
\newcommand{\defeq}{\coloneqq}

\newcommand{\emphc}[1]{\textcolor{blue}{#1}}
\newcommand{\emphcc}[1]{\textcolor{magenta}{#1}}

\newtheorem{theorem}{Théorème}
\newtheorem{proposition}[theorem]{Proposition}
\newtheorem{corollaire}[theorem]{Corollaire}

\begin{document}

Considérons le programme ci-dessous. Celui-ci cherche à déterminer si
un nombre $x$ est premier en essayant tous les diviseurs possibles de
$2$ jusqu'à $\lfloor \sqrt{x} \rfloor$. Nous allons démontrer
formellement que le programme est correct malgré l'usage risqué de
nombres en virgule flottante.

\begin{code}
  \inputminted{cpp}{code.cpp}
\end{code}

Dans le reste du document, nous faisons ces hypothèses standards:

\smallskip
\begin{itemize}
  \setlength\itemsep{3pt}

\item \og \mintinline{cpp}{sqrt} \fg{} est conforme à la norme
  IEEE 754;

\item les valeurs de type \mintinline{cpp}{double} sont représentées
  en nombre en virgule flottante double précision binaire de la norme
  IEEE 754, avec le mode d'arrondi qui approxime au nombre le plus
  près, où les bris d'égalité se font vers le nombre dont le dernier
  bit est pair.
\end{itemize}

\section{Borne sur la racine carrée}

Rappelons, qu'étant donné $x \in \R \setminus \{0\}$, nous écrivons
$\overline{x}$ afin de dénoter le nombre en virgule flottante
normalisé arrondi au nombre le plus près de $x$, et où un bris
d'égalité se fait vers le nombre dont le dernier bit est
pair. Rappelons également que l'erreur relative de $x$ est définie par
$\mathrm{err}(x) \defeq (x - \overline{x}) / x$.

Rappelons que les \mintinline{cpp}{double} ont une mantisse de $53$
bits et que leurs nombres normalisés appartiennent au domaine $\R'
\defeq \{x \in \R \setminus \{0\} : 2^{-1022} \leq |x| \leq 2^{1024} -
2^{971}\}$. La borne suivante sur l'erreur relative est bien connue.

\begin{proposition}\label{prop:err}
  $|\mathrm{err}(x)| \leq 2^{-53}$ pour tout $x \in \R'$.
\end{proposition}

Celle-ci permet de faire l'observation ci-dessous.

\begin{corollaire}\label{cor:err}
  Soient $n \in \N$ et $x \in \R' \cap [2^n, 2^{n+1})$. Nous avons $|x
    - \overline{x}| < 2^{n - 52}$.
\end{corollaire}

\begin{proof}
  Par la \cref{prop:err}, nous avons $|(x - \overline{x}) / x| \leq
  2^{-53}$. Puisque $x \geq 1$, cela implique que $|x - \overline{x}|
  \leq x \cdot 2^{-53}$. Comme $x < 2^{n+1}$, on en conclue que $|x -
  \overline{x}| < 2^{n - 52}$.
\end{proof}

Établissons maintenant une borne sur l'écart entre la racine carrée de
$x$ et la racine carrée de son approximation.

\begin{proposition}\label{prop:borne}
  Soient $n \in \N$ et $x \in \R' \cap [2^n, 2^{n+1})$. Nous avons
    $|\sqrt{x} - \sqrt{\overline{x}}| < 2^{\lfloor n/2 \rfloor - 52}$.
\end{proposition}

\begin{proof}
  Nous avons
  \begin{align*}
    \left|\sqrt{x} - \sqrt{\overline{x}}\,\right|
    %
    &= \frac{|x - \overline{x}|}{\sqrt{x} + \sqrt{\overline{x}}}
    && \text{(car $(a - b)(a + b) = a^2 - b^2$)} \\[5pt]
    %
    &< \frac{2^{n-52}}{\sqrt{x} + \sqrt{\overline{x}}}
    && \text{(par le \cref{cor:err})} \\[5pt]
    %
    &\leq \frac{2^{n-52}}{\sqrt{x} + \sqrt{x - 2^{n - 52}}}
    && \text{(car $\overline{x} \geq x - 2^{n - 52}$ par le \cref{cor:err})} \\[5pt]
    %
    &\leq \frac{2^{n-52}}{\sqrt{2^n} + \sqrt{2^n - 2^{n - 52}}}
    && \text{(car $x \geq 2^n$)} \\[5pt]
    %
    &\leq \frac{2^{n-52}}{\sqrt{2^n} + \sqrt{2^{n-1}}}
    && \text{(car $2^n - 2^{n - 52} \geq 2^{n-1}$)} \\[5pt]
    %
    &\leq \frac{2^{n-52}}{2 \cdot \sqrt{2^{n-1}}}
    && \text{(car $2^n \geq 2^{n-1}$)} \\[5pt]
    %
    &= \frac{2^{n-52}}{2^{(n+1)/2}} \\[5pt]
    %
    &\leq \frac{2^{n-52}}{2^{\lceil n /2 \rceil}} \\[5pt]
    %
    &= 2^{\lfloor n /2 \rfloor - 52}.
    && \qedhere
  \end{align*}
\end{proof}

\begin{theorem}\label{thm:racine}
  Soit $n \in \N$ tel que $1 \leq \lfloor n / 2 \rfloor \leq 51$ et
  soit $x \in \R' \cap [2^n, 2^{n+1})$. Nous avons
    $\overline{\sqrt{\overline{x}}} \geq \lfloor\sqrt{x}\rfloor$.
\end{theorem}
  
\begin{proof}
  Par définition, $\overline{x}$ se représente exactement en nombre en
  virgule flottante double précision. Ainsi, selon la norme IEEE 754,
  $\overline{\sqrt{\overline{x}}}$ est égal au nombre que l'on obtient
  en calculant d'abord $\sqrt{\overline{x}}$ avec précision infinie,
  puis en l'arrondissant.

  Remarquons que $\sqrt{x} \geq \sqrt{2^n} = 2^{n/2} \geq 2^{\lfloor
    n/2 \rfloor}$ et $\sqrt{x} < \sqrt{2^{n+1}} = 2^{(n+1)/2} \leq
  2^{\lfloor n/2 \rfloor + 1}$. Ainsi, en représentation en virgule
  flottante infinie (binaire), $\sqrt{x}$ peut s'écrire de cette
  forme:
  \begin{align}
  1{,}d_1 d_2 \cdots \times 2^{\lfloor n/2 \rfloor},\label{eq:forme1}
  \end{align}
  où $d_i = 0$ pour une infinité d'indices $i$.

  Par la \cref{prop:borne}, nous avons $|\sqrt{x} -
  \sqrt{\overline{x}}| < 2^{\lfloor n/2 \rfloor - 52}$. Ainsi, si
  $\sqrt{\overline{x}} - \sqrt{x}$ est négative, alors sa plus petite
  valeur possible peut s'écrire de cette forme en repré\-sen\-ta\-tion
  en virgule flottante infinie (binaire):
  \begin{align}
  -1{,}c_1 c_2 \cdots \times 2^{\lfloor n/2 \rfloor -
    53},\label{eq:forme2}
  \end{align}
  où $c_i = 0$ pour une infinité d'indices $i$.
  
  Mettons \eqref{eq:forme1} et \eqref{eq:forme2} sur un exposant
  commun:
  \begin{alignat}{3}
    \sqrt{x}:\qquad & &&1{,}\emphc{d_1 \cdots d_{\lfloor n/2 \rfloor}}
    \emphcc{d_{\lfloor n/2 \rfloor + 1} \cdots d_{52}}\ d_{53} d_{54} \cdots &{}
    \times 2^{\lfloor n/2 \rfloor}\label{eq:forme3}, \\
    %
    \sqrt{\overline{x}} - \sqrt{x}:\qquad &-&&0{,}\underbrace{\emphc{00
      \cdots 0000}\,\,}_{\text{$\lfloor n/2 \rfloor$~fois}}\underbrace{\emphcc{000000
      \cdots 0000}\,}_{\text{$52 - \lfloor n/2 \rfloor$~fois}} c_1 c_2 \cdots \cdots\ &{}
    \times 2^{\lfloor n/2 \rfloor}.\label{eq:forme4}
  \end{alignat}
  Notons que $\lfloor \sqrt{x} \rfloor = 1,\emphc{d_1 \cdots
    d_{\lfloor n/2 \rfloor}} \times 2^{\lfloor n/2 \rfloor}$. La somme
  de \eqref{eq:forme3} et \eqref{eq:forme4} est
  $\sqrt{\overline{x}}$. Si la soustraction sous-jacente n'emprunte
  aucun bit à $\emphc{d_1 \cdots d_{\lfloor n/2 \rfloor}}$, alors
  l'arrondi de $\sqrt{\overline{x}}$ est forcément supérieur ou égal à
  $\lfloor \sqrt{x} \rfloor$, et nous avons ainsi terminé. Supposons
  donc qu'il y a emprunt. Le résultat est de cette forme:
  \[
  \sqrt{\overline{x}}:\qquad
  %
  1,\emphc{d_1 \cdots d_i\ 0\ 1 \cdots 1}\ \emphcc{1 \cdots
    1}\ 1\ b_{54} b_{55} \cdots \times 2^{\lfloor n/2 \rfloor}.
  \]
  Puisque le \emphcc{$52^\text{ème}$ bit} après la virgule est impair,
  on arrondit forcément vers le haut (même si $b_{54} = b_{55} =
  \cdots = 0$):
  \[
  \overline{\sqrt{\overline{x}}}:\qquad
  %
  1,\emphc{d_1 \cdots d_i\ 1\ 0 \cdots 0}\ \emphcc{0 \cdots
    0} \times 2^{\lfloor n/2 \rfloor}.
  \]
  Par conséquent, $\overline{\sqrt{\overline{x}}} = \lfloor \sqrt{x}
  \rfloor$.
\end{proof}

\section{Le programme est correct}

Nous faisons une dernière observation (bien connue), puis nous
démontrons que le programme est correct.

\begin{proposition}\label{prop:premier}
  Soit $x \in \N_{\geq 2}$ un nombre qui n'est pas premier. Il existe
  $d \in \N$ tel que $2 \leq d \leq \lfloor\sqrt{x}\rfloor$ et $d$
  divise $x$.
\end{proposition}

\begin{proof}
  Comme $x$ n'est pas premier, il possède au moins deux facteurs
  premiers. Afin d'obtenir une contradiction, supposons que tous les
  facteurs premiers de $x$ sont strictement supérieurs à
  $\lfloor\sqrt{x}\rfloor$. Soient $p_1, p_2, \ldots, p_k$ les
  facteurs premiers de $x$. Par hypothèse, nous avons $p_i > \lfloor
  \sqrt{x} \rfloor$ pour tout $i \in [1..k]$. Comme chaque $p_i$ est
  entier, nous avons forcément $p_i > \sqrt{x}$ pour tout $i \in
  [1..k]$.

  Ainsi $x = p_1 p_2 \cdots p_k \geq p_1 p_2 > \sqrt{x} \cdot \sqrt{x}
  = x$, ce qui est une contradiction.
\end{proof}

\begin{theorem}
  Le programme est correct.
\end{theorem}

\begin{proof}
  Soit $x \in \N$ une entrée du programme. Comme $x$ est un entier non
  signé de 64 bits, nous avons $0 \leq x < 2^{64}$. Si $x \in \{0,
  1\}$, alors clairement le programme retourne le bon
  résultat. Supposons donc que $x \geq 2$. Par la \cref{prop:premier},
  l'algorithme utilisé par le programme est correct. En effet, si $x$
  est premier, alors il ne possède aucun diviseur, et si $x$ est
  premier alors il possède au moins un diviseur $d$ tel que $d \leq
  \lfloor \sqrt{x} \rfloor \leq \sqrt{x}$. Néanmoins, ce raisonnement
  suppose que toutes les opérations sont implémentées de façon
  exactes, ce qui n'est pas le cas. Il y a trois opérations risquées:

  \smallskip
  \begin{itemize}
    \setlength\itemsep{3pt}
    
  \item \og \mintinline{cpp}{double y = x;} \fg{} effectue une
    conversion qui approxime \texttt{x};

  \item \og \mintinline{cpp}{double z = sqrt(y);} \fg{} calcule une
    approximation de la racine carrée de \texttt{y}, qui elle-même est
    une approximation de \texttt{x};

  \item \og \mintinline{cpp}{d <= z} \fg{} compare un entier à un
    nombre en virgule flottante, bien que \texttt{d} soit possiblement
    trop grand pour être représenté exactement.
  \end{itemize}
  \smallskip

  Montrons que ces trois points ne sont pas problématiques. Supposons
  que $x$ n'est pas premier (car autrement il n'y a pas de
  problème). Nous considérons deux cas. Avant de procéder, remarquons
  que $\lfloor\sqrt{x}\rfloor$ est représentable exactement dans un
  \mintinline{cpp}{double} car
  \[
  \lfloor\sqrt{x}\rfloor
  %
  \leq \sqrt{x}
  %
  < \sqrt{2^{64}}
  %
  = 2^{32}
  %
  \leq 2^{53}.
  \]

  \medskip\noindent\emph{Cas 1: $2 \leq x \leq 2^{53}$}. Comme $x \leq
  2^{53}$, $x$ est représentable exactement dans un
  \mintinline{cpp}{double}. Autrement dit, $\overline{x} = x$. Par la
  \cref{prop:premier}, le plus petit facteur premier $p$ de $x$
  satisfait $p \leq \lfloor \sqrt{x} \rfloor$. Ainsi, $p$ est
  représentable exactement dans un \mintinline{cpp}{double}.

  Selon la norme IEEE 754, $\overline{\sqrt{\overline{x}}}$ est égal
  au nombre que l'on obtient en calculant d'abord
  $\sqrt{\overline{x}}$ avec précision infinie, puis en
  l'arrondissant. Nous avons $p \leq \lfloor \sqrt{x} \rfloor \leq
  \sqrt{x} = \sqrt{\overline{x}}$. Ainsi, $\sqrt{\overline{x}}$ n'est
  pas arrondi sous $p$. Plus formellement:
  \[
  p \leq \overline{\sqrt{\overline{x}}} = \texttt{z}.
  \]
  Puisque tous les entiers $\texttt{d} \in [2..p]$ sont représentables
  exactement en \mintinline{cpp}{double}, la condition \og
  \mintinline{cpp}{d <= z} \fg{} est évaluée correctement jusqu'à ce
  qu'on identifie correctement que $\texttt{d} = p$ est un diviseur.

  \bigskip\noindent\emph{Cas 2: $2^{53} < x < 2^{64}$}. Il existe un
  entier $n \in [53..63]$ tel que $2^n \leq x < 2^{n+1}$. Ainsi, nous
  avons $x \in \R'$ et clairement $1 \leq \lfloor n / 2 \rfloor \leq
  51$. Par conséquent, nous pouvons invoquer le
  \cref{thm:racine}. Nous avons donc
  \[
  \mintinline{cpp}{z} = \overline{\sqrt{\overline{x}}} \geq
  \lfloor\sqrt{x}\rfloor.
  \]
  Ainsi, la boucle \mintinline{cpp}{while} considère tous les
  diviseurs \texttt{d} tels que $\texttt{d} \leq
  \lfloor\sqrt{x}\rfloor$, pourvu que la comparaison \og
  \mintinline{cpp}{d <= z} \fg{} soit correcte. Montrons que c'est le
  cas.

  Comme mentionné précédemment, $\lfloor\sqrt{x}\rfloor$ est
  représentable exactement dans un \mintinline{cpp}{double}. Ainsi,
  toutes les comparaison sont exactes pour $\texttt{d} \in
  [2..\lfloor\sqrt{x}\rfloor]$.
\end{proof}

\end{document}
